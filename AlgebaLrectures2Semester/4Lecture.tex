\bd{Гомоморфизм колец и идеал.}\\

Кольцои является группа по сложению и полу группа по умножению.\\
Гомоморфизм сохраняет обе операции.\\
Базовое в кольце это операция сложения и сначала надо изучить гомоморфизм
комутотивной группы по сложению.\\


\bd{Структура гомоморфизма группы.}\\

$\alpha : b \to H$ ядром гоморфизма $\alpha$ называется такое подмножество
группы $b ~ g\alpha = \{g \in G| \varphi (g) = \varphi_H$\\
Свойство 1:\\
Ядро явлется подгруппа.\\
Докажим это по критерию подгруппы. Пусть $g_1,g_1 \in Kep~\alpha
~~ g_1 \cdot g_2 \Rightarrow 1 \in Ker~\alpha$ $\varphi(g_1,g_2) =
\varphi (g_1)\varphi (g_2) = e_H \cdot e_H = e_H$\\
По определению обратого элемента. $g\cdot g^{-1} = 1$\\
$\varphi(e_G) = \varphi(e_G e_G) = \varphi(e_G) \cdot \varphi(e_G) \in H$\\
Так как $H$ группа то у элемента $\varphi(e_G)$ есть обратный, умножая на него
обе части равенства $\varphi = e_H$ таким образом еденичный переходит в
еденичный.\\
$e_H = \varphi(e_G) = \varphi(g\cdot g^{-1} = \varphi(g)\cdot \varphi(g^{-1}
\Rightarrow \varphi(g^{-1} = e_H g^{-1} \in Ker~\varphi$\\

Определение: Пусть $G$ группа $L$ подгруппы $G \ge L$\\
$\ge$ - означает число подмножеств согласованную структуру с объедененим.\\
Подгруппа называется нормальной если для $\forall g \in L ~ g^{-1} Lg \le L$\\
$g_{-1} lng = ln$ называется сопряженным элементом $H$ при помощи\\
Понятия ненормальной подгруппы нет.\\
Нормальная подгруппа выдерживает все сопряжение $G$ группе $G$.\\

Утверждение: Ядро является нормальной подгруппой.\\
Доказательство: Пусть $f \in Kerp~\varphi, g\in L$ проверим что $\varphi(g^{-1}
fg) = \varphi_H$\\
По определению гомоморфизма $\varphi(g^{-1} fg) = \varphi(g^{-1}) \cdot
\varphi(f) \cdot \varphi(g) = \varphi(g^{-1}) \cdot \varphi_H \cdot \varphi(g) =
\varphi(g^{-1}) \cdot \varphi(g) e_H$\\

Группа называется простой если в ней нет не правильных нормальных подгрупп.\\
Правильная это еденичная или вся группа.\\

Основная задача группы описывать все простые группы, а остальные группы получить
из расширения.\\

Бесконечные группы не описаны.\\

В теории конечных групп вроде решена. Было найденоо 17 бесконенчо серий простых
групп. 26 не серийных изолированных групп.\\\\

Проблема 4 красок: Ломанными линиями разделим плоскость на облости. Области
называеются соседними если у них есть общий кусок границы.\\
Задача: хватит ли 4 красок чтобы раскрасить всю карту чтобы соседние не были
одного цвета.\\
Эта задача с перебором 1500 графов была решена на компьютере.\\

\bd{Фактор группы, фактор кольца.}\\
треугольник - нормальная группа.\\
$G 	!!!!треугольник!!! H$ подмножество $gH = \{g\cdot h | ln \in H\}$ называется
смежными классами. Не сложно проверить, что разны разные смежные классы не
пересекаются.\\
$h \to gh$ - биекция.\\
На множество смежных классов мы введем уномжение $g_1H\cdot g_2H = (g_1g_2)H$\\

Определение: в смежных классах $G$ элементы $g$ называются представителями. Не
сложно проверить что в качестве представителя можно взять любой элемент этого
смежного класса.\\
Необходимо проверить коректность задания оперпцией так как она определена через
представителей. Убедимся в этом заменив представителей, мы получим тот же
результат.\\
$g_1' = g_1 h_1 \in H$\\
$g_2 H = g_2 H ~~~~ g_2' = g_2 h_2 \in H$\\
$
  (g_1' H)\cdot(g_2' H) - g\cdot g_2 H = (g_1 h_1)(g_2 h_2)H =
  (g_1 \cdot g_2)\cdot(g_{1}^{-1} \cdot hg_2)h_2 H = g_1 g_2
$\\
Так как подгруппа $H$ нормальная то $g_1 h_1 = g_{2}^{-1} \in H$\\
Не трудно проверить $\forall h\in H h\cdot H = N$\\

Определение: группа элементов которого является смежные классы операция задана
как указано выше называется группой фактор и обозначается $G/H$.\\
Нейтральным элементом в этой группе $e\cdot H = H$\\
Обратный $(gH)^{-1} = g^{-1}H$\\
Ассоциативность следует из ассоциативности\\
Умножение из $G$\\
Когда мы число множество составленный как еденичный объект это называется фактор
множества, стандартная идея при обобщении идеи.\\
Пусть $L$ группа $H$ нормальная подгруппа ($L(H)$ - фактор группы) говорят что
группа $G$ является ывлфаовыж группы $H$ при вафвыа группы $L$. В конце концов
мы можем свести к простым группам, где строить фактор группы не получится.\\

Даже зная все простые группы, построить все группы крайне затруднительно,
потому что способов расширения очень много.\\

\bd{Идеалы и факторы колец.}\\

Пусть $K$ кольцо, а $L$ его под кольцо и одновременно подгруппа его.\\
Подкольцо называется идеалом если для $\forall k \in K k\cdot L = \{k|e\in L\}
= lv = c$\\
$L$ - обобщение нуля.\\
Смежные классы по идеалу имеют вид $k\in K$. На множестве смежных классов введем
операцию сложения\\
$(k_1 + L_1) + (k_2 + L_1) = k_1 + k_2 + L_1$\\
$(k_1+L_1)\cdot(k_2+L_1) = k_1 k_2 + L_1$\\
Так как здесь операции задаются при помощи представителей, то нужно проверить
коректность.\\
$k_1' + L_1 = k_1 + L_1$\\
$k_2' + L_1 = k_2 + L_1$\\