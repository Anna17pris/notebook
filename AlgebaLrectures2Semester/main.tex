\documentclass{article}

\usepackage[utf8]{inputenc}
\usepackage{amsmath,amssymb}
\usepackage[russian]{babel}
\usepackage{pgfplots}
\pgfplotsset{compat=1.9}
%\usepackage[pdftex]{color} пакет pdfplots уже подключил пакет color
\usepackage{cancel}

\parindent=0pt
\newcommand{\bd}[1]{{\bfseries #1}} %текст жирным (bold)
\newcommand{\kv}{\emph}  %текст курсивом
\newcommand{\mt}{\mathit} %математический шрифт в формулах (mathText)
\newcommand{\bk}[1]{\bd{\kv{#1}}} %текст жирный и курсивным шрифтом

\newenvironment{theorem}{\colorbox[rgb]{1, 0.4, 0.6}{\bd{Теорема:}}\\}%
{\colorbox[rgb]{0, 0, 0}{1}\\} %Теорема

\newenvironment{proof}{\colorbox[rgb]{0, 0.8, 0.5}{\bd{Доказательство:}}\\}%
{\colorbox[rgb]{0, 0, 0}{1}\\} %Доказательство

\newenvironment{defin}{\colorbox[rgb]{1, 0.4, 1}{\bd{Определение:}}\\}%
{\colorbox[rgb]{0, 0, 0}{1}\\} %definition определение

\renewenvironment{title}{\begin{center} \bfseries \LARGE}{\end{center}}
%заголовок

\begin{document}

\begin{title}
  Математический анализ.\\
  (Второй семестр)
\end{title}

\begin{title}
  Иследование монотонности функции с помощью производной.
\end{title}

\begin{theorem}
  Пусть $f(x)$ - непрерывана на $<a; b>$ и имеет $f'(x)$ на $(a;b)$ то
  \\
  Если $\forall x \in (a,b) ~~ f'(x) \ge 0 $ то $f(x) \nearrow ~~ <a; b>$\\
  Если $\forall x \in (a,b) ~~ f'(x) \le 0 $ то $f(x) \searrow ~~ <a; b>$
\end{theorem}

\begin{proof}
  Пусть $\forall x_1, x_2 \in <a,b>$ и при этом $x_1 < x_2 $\\
  Запишем \kv{теорему Лагранжа о формуле конечных преращений.}\\
  $f(x_2) - f(x_1) = f'(c)(x_2 - x_1)$ очевидно что $x_2 - x_1 > 0 $\\
  Если $f'(c) \ge 0$ то $f(x_2) \ge f(x_1) \Rightarrow f(x)$ на $<a,b>$
  \kv{возрастает}.\\
  Если $f'(c) \le 0$ то $f(x_2) \le f(x_1) \Rightarrow f(x)$ на $<a,b>$
  \kv{убывает}.
\end{proof}

\begin{theorem}
  Пусть $f(x)$ - непрерывана на $<a; b>$ и имеет $f'(x)$ на $(a;b)$ то\\
  Если $\forall x \in (a, b) ~~ f'(x) > 0$ то $f(x)$ \kv{строго}
  $\nearrow (a; b)$\\
  Если $\forall x \in (a, b) ~~ f'(x) < 0$ то $f(x)$ \kv{строго}
  $\searrow (a; b)$
\end{theorem}

\begin{proof}
  Пусть $\forall x_1, x_2 \in <a,b>$ и при этом $x_1 < x_2 $\\
  Запишем \kv{теорему Лагранжа о формуле конечных преращений.}\\
  $f(x_2) - f(x_1) = f'(c)(x_2 - x_1)$ очевидно что $x_2 - x_1 > 0$\\
  Если $f'(c) > 0$ то $f(x_1) > f(x_2) \Rightarrow f(x)$ на $<a,b>$
  \kv{возрастает}.\\
  Если $f'(c) < 0$ то $f(x_2) < f(x_1) \Rightarrow f(x)$ на $<a,b>$
  \kv{убывает}.
\end{proof}

\begin{title}
  Исследование функции на экстремум с помощью производной.
\end{title}

\bd{Необходимые и достаточные условия экстремума.}\\
\begin{theorem}
  Если: \\
    1) $f(x)$ дифференцируема в проколотой окрестности точки $a$.\\
    2) Непрерывна в точке $a$.\\
    3) При переходе через точку $a$ $f(x)$ меняется с - на +. То есть\\
      $\exists \delta > 0 ~~ \forall x \in (a - \delta; a) ~~ f'(x) < 0$\\
      $\exists \delta > 0 ~~ \forall x \in (a; a + \delta) ~~ f'(x) > 0$\\
  Тогда $a$ - точка \kv{строгого минимума.}
\end{theorem}

\begin{proof}
  По формуле конечных приращений\\
  $\forall x \in (a - \delta; a) ~~ f(x) - f(a) = f'(c)(x - a)$ очевидно что
  $x - a < 0$ и \\ $f'(c) < 0$ $\Rightarrow f(x) > f(a)$\\
  $\forall x \in (a; a + \delta) ~~ f(x) - f(a) = f'(c)(x - a)$ очевидно что
  $x - a > 0$ и \\ $f'(c) > 0$ $\Rightarrow f(x) > f(a)$\\
  $\forall x \in (a - \delta; a+ \delta)\backslash \{a\} ~~ f(x) > f(a)$\\
  Тогда $a$ - точка \kv{строгого минимума.}
\end{proof}

\begin{theorem}
  Если: \\
    1) $f(x)$ дифференцируема в проколотой окрестности точки $a$.\\
    2) Непрерывна в точке $a$.\\
    3) При переходе через точку $a$ $f(x)$ меняется с + на -. То есть\\
      $\exists \delta > 0 ~~ \forall x \in (a - \delta; a) ~~ f'(x) > 0$\\
      $\exists \delta > 0 ~~ \forall x \in (a; a + \delta) ~~ f'(x) < 0$\\
  Тогда $a$ - точка \kv{строгого максимума.}
\end{theorem}

\begin{proof}
  По формуле конечных приращений\\
  $\forall x \in (a - \delta; a) ~~ f(x) - f(a) = f'(c)(x - a)$ очевидно что
  $x - a < 0$ и \\ $f'(c) > 0$ $\Rightarrow f(x) < f(a)$\\
  $\forall x \in (a; a + \delta) ~~ f(x) - f(a) = f'(c)(x - a)$ очевидно что
  $x - a > 0$ и \\ $f'(c) < 0$ $\Rightarrow f(x) < f(a)$\\
  $\forall x \in (a - \delta; a+ \delta)\backslash \{a\} ~~ f(x) < f(a)$\\
  Тогда $a$ - точка \kv{строгого максимума.}
\end{proof}

\begin{theorem}
  Пусть $f(x)$ в точке $a$\\
  $f^{(n)} (a) \neq 0 ~~ n > 1\\
  f^{(k)} (a) = 0 ~~ k = 1, 2, \ldots, n-1$\\
  Если $n$ - не четный, то в точке $a$ - нет экстремума.\\
  Если $n$ - четный, то $a$ - точка экстремума.\\
  При чем если $f^{(n)} (a) > 0$ то $a$ - точка строгого минимума.\\
  Если  $f^{(n)} (a) < 0$ то $a$ - точка строгого максимума.
\end{theorem}

\begin{proof}
  Формула Тейлора ипользуется для доказательства в форме Пиано:\\
  $f(x) - f(a) = \frac{f^{(n)}(a)}{n!} \cdot (x - a)^n + \alpha (x)
  \cdot (x - a)^n = (x - a)^n \left(\frac{f^{(n)}(a)}{n!} + \alpha(x) \right)\\
  \alpha (x) \to 0 ~~ x \to a$ \\
  Выберем $\delta$ так, чтобы $\delta > 0 ~~ \forall x \in
  (a - \delta; a + \delta) > |d(x)| < \frac{f^{(n)}(a)}{2n!}\\
  \sign \left(\frac{f^{(n)}(a)}{n!} + \alpha(x)\right) = \sign (f^{(n)}(a)$\\
  Пусть $r$ - не четный. Если $(x - a)^n$ - меняет знак, то и
  f(x) - f(a) - меняет знак и нет экстремумума\\
  Пусть $n$ - четный $f(x) - f(a) > 0$, то $a$ - точка строгого минимума.\\
  Пусть $n$ - четный $f(x) - f(a) < 0$, то $a$ - точка строгого максимума.
\end{proof}

\bd{Алгоритм нахождения наибольшего и наименьшего значения функции.}\\
Дана функция $f(x)$ на $[a;b]$. Необходимо\\
1) Найти все критические точки $x_1, x_2, \ldots, x_n \in [a; b]$\\
2) Вычислить значения критических точек и крайних точек отрезка:\\
$f(a), f(b), f(x_1), f(x_2), \ldots, f(x_k) ~~ k = 1, 2 \ldots n$\\
$max \{f(a); f(b); f(x_k), k = 1, 2, \ldots, n\} = c - $ тогда \kv{наибольший}\\
$min \{f(a); f(b); f(x_k), k = 1, 2, \ldots, n\} = c - $ тогда \kv{наименьший}\\
%================================Заголовок=================
\begin{title}
	Поля разложения.
\end{title}

\kv{Многочлен называется неприводимым} над полем P если он
не раскладывается на произведение многочленов меньших
степеней.\\

\kv{Неприводимый многочлен} это простой элемент кольца
многочлена.\\

Простые чила неприводимые многочлены.\\

Нужно построить такое поле F в котором многочлен
разлогается на линейные множетели и называется F полем
разложения.

$Q_f = \{g(x) | g(x)_{\mt{ост}}\subset f(x)_{\mt{ост}}\}$

$Z_n$ идейно делени на многочлен.\\

На множестве $Q_f$ заданы операции сложения и умножения.
Сложение это обычное сложение многочленов. Оперция
умножения обычное умножениу, а потом деление с остатком
на $f(x)$  так же как и а кольце $Z_n$  проверяется $Q_f$
это комутативное кольцо $1\in P ~  0\in P$\\

Проверим что $Q_f$ поле:\\
Как и в случае $Z_p$ произвольный многочлен
НОД$(g(x),f(x)) = 1$ По следствию из алгоритма Евклида
$\exists u(x),v(x) ~ u(x)g(x) + v(x)f(x) = 1 ~
u(x)g(x) = 1 - v(x)f(x)$ значит остаток от деления на
$f(x)$ произведение $u(x)g(x)=1 u(x)=g(x)_1 ~ Q_f\supset P$
корень $f(x) = 0$.\\

1 шаг. Различие для множества $f(x)\in f(x)P[x]$ x-это
просто символ для записи многочлена, а в $f(x) = 0$ х
это отсаток принадлежащий $x\in Q_f$. Из чисто методических
соображений это вновь приобретенный корень смыслах
$
x\in Q_t$ мы обозначим $\alpha$, тогда  $f(\alpha) = 0
Q_f = \{g(\alpha) |
g(\alpha)_{\mt{ост}}\subset f(\alpha)_{\mt{ост}}\}
$\\

2 шаг. $Q_f f(x) = Q_f[x] ~
\exists (x-\alpha)_{\alpha}\cdot g(x)$ $g(x)$ разлогаем
на неприводимые множетели и к этим множетелям применяем
шаг 1.\\
Мы добавим корни $\beta \gamma$ и так далее. В конце концов
$f(x)$ разложится на множетели.\\

Пример: $P\in R f(x) = x^2 + 1$\\
Неимеет корней значит неприводима.\\
$Q_{x^2+1} = \{a, ax + b | a,b\in R\}$\\
Чтобы не менять традицию пое остатков будем обозначать
вместо х символ i.\\
$Q_{x^2+1} = \{a, ax + b | a,b\in C\}$\\
$(a + ib)(a_1 + ib_1) =
a{a_1} + i(a{b_1} + b{a_1}) + ib{b_1}$\\
для поле разложения $x^2 + 1$ есть специальное обозначение
$C = \{a + bi | i^2 = -1 a,b\in R\}$\\
C расматривается как вектороное пространство над полем $P$,
имеет размерность ююююююю и обозначают\\
$Z = ai + b ~ a\in R$\\
$Z_1 = ai + b Z + Z_1 = (a + a_1) + i(b_1 + b)$\\
Если мнимой части меняет знак то получится так называемое
сопряженное число.
$Z + Z' = 2a\in R$\\
$Z \cdot Z' = (a + bi)(a - bi) = a^2 - {i^2}{b^2} =
a^2 + b^2 ~ \in P$\\
$|Z| = \sqrt{Z\cdot Z'} = \sqrt{a^2 + b^2}$\\
$(Z\cdot Z)' = Z' \cdot Z'$\\
$(Z + Z)' = Z' + Z'$\\

Нахождение обратного элемента\\
НОД$a+ib ~~ i^2 + 1 = 1$\\

$Z \cdot Z' = a^2 + b^2$ тогда
$\frac{Z \cdot Z'}{a^2 + b^2} = 1 \Rightarrow
Z^{-1} = \frac{Z'}{Z\cdot Z'}$\\


\begin{center}
  \tit{Интегрирование функции одной переменной.}
\end{center}

\tit{Первообразная и неопределенный интеграл. Свойства:}\\
Пусть $F(x); f(x)$ пределена на $<a;b>$ F - непрерывна на $<a,b>$,
дифференцируема на $(a;b)$ и $\forall x \in (a; b) ~~ F'(x) = f(x)$, то $F(x)$
называют первообразной для $f(x)$ на $(a;b)$.\\

\bd{Свойства:}\\
1) Если F(x) - первообразная для $f(x)$ определена на $\Delta$ то $F(x) + C$
также первообразная к $f(x)$.\\
\begin{proof}
  $(F(x) + C)' = F'(x) + 0 = f(x)$.\\
\end{proof}

2) Если $F_{1}(x), F_{2}(x)$ обе первообразнозные для $f(x)$ на промежутке
$\Delta$, то $F_{1}(x) - F_{2}(x) = C$.\\
\begin{proof}
  $(F_{1}(x) - F_{2}(x))' = F'_{1}(x) - F'_{2}(x) = f(x) - f(x) \equiv 0$\\
  $F_{1}(x) - F_{2}(x) = const$.\\
\end{proof}

\begin{defin}
  Совокупность всех первообразных для $f(x)$ на промежутке $\Delta$ называют
  неопределенным интегралом от $f(x)$ и обозначают
  \[\int f(x)dx\]
  $f(x)$ - называют подынтервальной функцией.\\
  $f(x)dx$ - Подинтегральным выражением.
  \[\int f(x)dx = \not\{ F(x) + C \not\} \]
  Где $F(x) + C$ - семейство функций.\\
\end{defin}

1) Диффиренциал от неопределенного интеграла равен подинтергальному выражению:
\[d \left ( \int f(x)dx \right ) = f(x)dx \]
\begin{proof}
  \[d \left ( \int f(x)dx \right ) = d(F(x) + C) = dF(x) + 0 = F'(x)dx
  = f(x)dx\]
\end{proof}

2)\[\int dF(x) = F(x) + C\]
\begin{proof}
  $\int dF(x) = \int F'(x)dx = F(x) + C$ \\
\end{proof}

3) $a, b \in R$\\
\[\int (af(x) + bf(x)) = a \cdot \int f(x)dx + b \cdot \int g(x)dx\]
\begin{proof}
  $aF(x) + bG(x) \in \int (af(x) + bg(x))dx\\
  aF'(x) + G'(x) = af(x) + bg(x)$ \\
\end{proof}

\begin{center}
  \tit{Замена переменных (подстановка) в неопределенных интегралах.}\\
\end{center}

\begin{theorem}
  Пусть $F(t)$ - первообразная для $f(x)$ на промежутке $T$ \\
  $t = \phi (x)$ непрерывна и дифференциируема на $\phi(\delta) \subset T$ то\\
  \[\int f(\phi (x)) \phi'(x)dx = F(\phi (x)) + C\]
\end{theorem}

\begin{proof}
  \[(F(\phi(x)) + C)' = F'(\phi (x)) \cdot \phi'(x) + 0 = f(\phi(x))
  \phi'(x)\]\\
\end{proof}
\bd{Гомоморфизм колец и идеал.}\\

Кольцои является группа по сложению и полу группа по умножению.\\
Гомоморфизм сохраняет обе операции.\\
Базовое в кольце это операция сложения и сначала надо изучить гомоморфизм
комутотивной группы по сложению.\\


\bd{Структура гомоморфизма группы.}\\

$\alpha : b \to H$ ядром гоморфизма $\alpha$ называется такое подмножество
группы $b ~ g\alpha = \{g \in G| \varphi (g) = \varphi_H$\\
Свойство 1:\\
Ядро явлется подгруппа.\\
Докажим это по критерию подгруппы. Пусть $g_1,g_1 \in Kep~\alpha
~~ g_1 \cdot g_2 \Rightarrow 1 \in Ker~\alpha$ $\varphi(g_1,g_2) =
\varphi (g_1)\varphi (g_2) = e_H \cdot e_H = e_H$\\
По определению обратого элемента. $g\cdot g^{-1} = 1$\\
$\varphi(e_G) = \varphi(e_G e_G) = \varphi(e_G) \cdot \varphi(e_G) \in H$\\
Так как $H$ группа то у элемента $\varphi(e_G)$ есть обратный, умножая на него
обе части равенства $\varphi = e_H$ таким образом еденичный переходит в
еденичный.\\
$e_H = \varphi(e_G) = \varphi(g\cdot g^{-1} = \varphi(g)\cdot \varphi(g^{-1}
\Rightarrow \varphi(g^{-1} = e_H g^{-1} \in Ker~\varphi$\\

Определение: Пусть $G$ группа $L$ подгруппы $G \ge L$\\
$\ge$ - означает число подмножеств согласованную структуру с объедененим.\\
Подгруппа называется нормальной если для $\forall g \in L ~ g^{-1} Lg \le L$\\
$g_{-1} lng = ln$ называется сопряженным элементом $H$ при помощи\\
Понятия ненормальной подгруппы нет.\\
Нормальная подгруппа выдерживает все сопряжение $G$ группе $G$.\\

Утверждение: Ядро является нормальной подгруппой.\\
Доказательство: Пусть $f \in Kerp~\varphi, g\in L$ проверим что $\varphi(g^{-1}
fg) = \varphi_H$\\
По определению гомоморфизма $\varphi(g^{-1} fg) = \varphi(g^{-1}) \cdot
\varphi(f) \cdot \varphi(g) = \varphi(g^{-1}) \cdot \varphi_H \cdot \varphi(g) =
\varphi(g^{-1}) \cdot \varphi(g) e_H$\\

Группа называется простой если в ней нет не правильных нормальных подгрупп.\\
Правильная это еденичная или вся группа.\\

Основная задача группы описывать все простые группы, а остальные группы получить
из расширения.\\

Бесконечные группы не описаны.\\

В теории конечных групп вроде решена. Было найденоо 17 бесконенчо серий простых
групп. 26 не серийных изолированных групп.\\\\

Проблема 4 красок: Ломанными линиями разделим плоскость на облости. Области
называеются соседними если у них есть общий кусок границы.\\
Задача: хватит ли 4 красок чтобы раскрасить всю карту чтобы соседние не были
одного цвета.\\
Эта задача с перебором 1500 графов была решена на компьютере.\\

\bd{Фактор группы, фактор кольца.}\\
треугольник - нормальная группа.\\
$G 	!!!!треугольник!!! H$ подмножество $gH = \{g\cdot h | ln \in H\}$ называется
смежными классами. Не сложно проверить, что разны разные смежные классы не
пересекаются.\\
$h \to gh$ - биекция.\\
На множество смежных классов мы введем уномжение $g_1H\cdot g_2H = (g_1g_2)H$\\

Определение: в смежных классах $G$ элементы $g$ называются представителями. Не
сложно проверить что в качестве представителя можно взять любой элемент этого
смежного класса.\\
Необходимо проверить коректность задания оперпцией так как она определена через
представителей. Убедимся в этом заменив представителей, мы получим тот же
результат.\\
$g_1' = g_1 h_1 \in H$\\
$g_2 H = g_2 H ~~~~ g_2' = g_2 h_2 \in H$\\
$
  (g_1' H)\cdot(g_2' H) - g\cdot g_2 H = (g_1 h_1)(g_2 h_2)H =
  (g_1 \cdot g_2)\cdot(g_{1}^{-1} \cdot hg_2)h_2 H = g_1 g_2
$\\
Так как подгруппа $H$ нормальная то $g_1 h_1 = g_{2}^{-1} \in H$\\
Не трудно проверить $\forall h\in H h\cdot H = N$\\

Определение: группа элементов которого является смежные классы операция задана
как указано выше называется группой фактор и обозначается $G/H$.\\
Нейтральным элементом в этой группе $e\cdot H = H$\\
Обратный $(gH)^{-1} = g^{-1}H$\\
Ассоциативность следует из ассоциативности\\
Умножение из $G$\\
Когда мы число множество составленный как еденичный объект это называется фактор
множества, стандартная идея при обобщении идеи.\\
Пусть $L$ группа $H$ нормальная подгруппа ($L(H)$ - фактор группы) говорят что
группа $G$ является ывлфаовыж группы $H$ при вафвыа группы $L$. В конце концов
мы можем свести к простым группам, где строить фактор группы не получится.\\

Даже зная все простые группы, построить все группы крайне затруднительно,
потому что способов расширения очень много.\\

\bd{Идеалы и факторы колец.}\\

Пусть $K$ кольцо, а $L$ его под кольцо и одновременно подгруппа его.\\
Подкольцо называется идеалом если для $\forall k \in K k\cdot L = \{k|e\in L\}
= lv = c$\\
$L$ - обобщение нуля.\\
Смежные классы по идеалу имеют вид $k\in K$. На множестве смежных классов введем
операцию сложения\\
$(k_1 + L_1) + (k_2 + L_1) = k_1 + k_2 + L_1$\\
$(k_1+L_1)\cdot(k_2+L_1) = k_1 k_2 + L_1$\\
Так как здесь операции задаются при помощи представителей, то нужно проверить
коректность.\\
$k_1' + L_1 = k_1 + L_1$\\
$k_2' + L_1 = k_2 + L_1$\\
\begin{center}
  \bd{Гомоморфизм колец и Идеалов}
\end{center}

Кольцо является группой по сложению и полугруппой по умножению и гомоморфизм 
созраняет обе операции. \\
Базовая в кольце операция - это \kv{сложение}. И прежде всего надо изучить
гомоморфизм комутативной группы по сложению.\\

\bd{Структуры гомоморфизма групп}\\
$\varphi:G \to H$\\
Ядром гомоморфизма $\varphi$ -называют такое подмножество группы G, где 
$Ker \varphi = {g \in G|\varphi (g) = e_H}$ \kv{Ker - ядро}\\

\kv{Свойство 1}\\
Ядро - подгруппа.\\
Докажем по критерию подгруппы\\
$g_1, g_2, \in Ker\varphi, g^{-1}_1 \in Ker, ~~~ Ker \in 1$\\
$\varphi(g_1, g_2) = \varphi(g_1) \varphi(g_2) = e_H \cdot e_H = e_H$\\
По определению обратного элемента $g_1 \cdot g^{-1}_1 = e_G ~~~ 
\varphi(e_G) = \varphi(e_G e_G) = \varphi(e_G) \cdot \varphi(e_G)$\\

Так как \kv{H} - группа, то у элемена $\varphi(e_G)$ - есть обратный. Умножим 
на него обе части равенства $\varphi(e_G) = e_H$\\
Таким образом единичный элемент переходит в единичный\\
$e_H = g(e_G) = \varphi(g_1 g^{-1}_1) = \varphi(g_1)\varphi(g^{-1}_1) 
\Rightarrow \varphi(g^{-1}_1) = e_H$\\

\bd{Определение}\\
$G \ge L$ ~~~ $\ge$ - означает, что подмножество имеет согласованную структуру\\
Подгруппа нормальная, если $\forall g \in G ~~~ g^-1 Lg \subseteq L$\\
$G^{-1}hg - сопряжается элемент h при помощи элемента q$\\
\kv{Понятие не нормальной подгруппы нет. ее название подгруппа, не являющаяся 
нормальной}\\

\kv{Свойство 2}\\
Ядро - нормальная подгруппа.\\
Доказательство\\
Пусть $f \in Ker\varphi ~~~ \varphi(g^{-1} fg) = e_H ~~~ g\in G$\\
По определению гомоморфизма 
$\varphi(g^{-1} fg) = \varphi(g^{-1}) \varphi(f) \varphi(g) = \varphi(g)^{-1}
\varphi(f) \varphi(g) = \varphi(g)^{-1} e_H \varphi(g) = \varphi(g)^{-1} 
\varphi(g) = e_H$\\
Группа называется простой, если нет нетривиальной нормальной подгруппы. 
(Тривиальная - это группа, состоящая из одного элемента)\\

\kv{Основной задачей теории групп является описание всех простых групп, а 
остальные группы получаются из простых при помощи расширений.}\\
\kv{Бесконечные группы - не описаны.}\\
\kv{В теории конечных групп задача вроде решена (были найдены 17 бесконечных 
серий групп и 26 не серийных изолированных групп)}\\

\begin{center}
  \bd{Фактор группы Фактор кольца}
\end{center}

$G \ge H$ - H нормальная подгруппа. Подмножество вида $gH = {gh | h \in H}$
называется смежными классами.\\
\kv{Не сложно проверить, что разные смежные классы не пересекаются}\\ 
$h \mapsto gh$ - биекция.\\
$g_1H /cdot g_2H = (g_1 g_2)H$\\

\bd{Определение}\\

В смежных классах $gH$, элемент п - представитель. Не сложно проверить, что в 
качестве представителя можно взять любой элемент этого смежного класса. 
Необходимо проверить корректность задания операции, так как она определяется 
через представителей убедимся, что заменив представителей, мы получим тот же 
результат.\\
$g_1H = g'_1H\\
g'_1 = g_1 h_1 \in H\\
g_2H = g'_2H\\
g'_2 h_2 \in H\\
h \mapsto gh ~~~ (g'_1 H)(g'_2 H) = (g'_1 g'_2)H = (g_1 h_1)(g_2 h_2)H = 
(g_1 g_2)(g'_2 h_1 g_2)h_2 H = g_1 g_2 H$\\
Так как подгруппа H - нормальная, то $g^{-1}_2 h_1 g_2 \in H = h_3 ~~~ 
\forall h \in H ~~~ h \cdot H = H$\\

\bd{Определение}\\
Группа элементы, которой являются смежными классами, определенно задана как 
указано выше и называется фактор группы - $G/H$\\
Нейтральный элемент в этой группе $e \cdot H = H$\\
Обратный смежный класс $(gH)^{-1} = g^{-1}H$ ассоциотивность следует из 
ассоциотивности умножения.\\
Когда целове множество воспринимаем как единый объект - оно называется фактором 
множества - стандартная идея при обобщении.\\
$G/H = L$ Пусть $G$ - группа, $H$ - нормальная подгруппа, $G/H = L$ - фактор 
группы. Говорят что группа $b$ расширение группы $H$ при помощи группы С.\\
В конце концов можно свести к простым группам, где строить факторы группы не 
получится. Даже зная все простые группы - все группы построить затруднительно
так как способов расширений - бесконечное число.\\

\begin{center}
  \bd{Идеалы и фактор кольца}
\end{center}

Пусть $K$ - кольцо, $L$ - подкольцо и подгруппа по сложению. Подкольцо 
называется идеалом, если $\forall k \in K ~~~ k \cdot L = {kl | l \in L} = 
Lk = L$\\
\bk{Идеал} - обобщение понятия нуля.\\
Смежные класс по идеалу имеют вид $k + L, k \in K$. На множестве смежных классов
введем операцию сложения и умножения\\
$(k_1 + L) + (k_2 + L) = (k_1 + k_2) + L\\
(k_1 + L) \cdot (k_2 + L) = (k_1 \ cdot k_2) + L$\\
Операции задаются при помощи представителей - проверим их корректность.\\
$k'_1 + L = K_1 + L\\
k'_2 + L = K_2 + L ~~~ I \triangle K ~~~ IK \le I ~~~ I + I = I$\\
Если $K$ - комутативное кольцо, то существуют левые идеалы и правые, а также 
двусторонние.\\

\bd{Определение}\\
Идеал называется максимальным, если он не содержится не в каком больше. 
В дальнейшем будем будем считать, что кольцо комутативное $K + I = {K + i 
| i \in I}$\\
А множество всех счетных классов заданы сложением и умножением и называются 
фактор-кольцом - $K/I$\\

\kv{Самые полезные идеалы - главые. Главные идеалы - пораждаются одним элементом
который является главным}\\

\bd{Теорема}\\
Любое Евклидово кольцо - кольцо главных идеалов.\\
\bd{Доказательство}\\
Пусть есть произвольный идеал $I = {i_1, i_2, ... i_n}$ Надо выбрать такой 
элемент $i$, через который можно выразить все остальные $i$. Так как кольцо 
Евклидово, то в нем любые 2 элемента имеют Наибольций Общий Делитель.\\
$d = НОД(i_1; i_2) ~~~ i_1 d \cdot i'_1 ~~~ i_2 = d \cdot i'_2$\\
Таким образом и $i_1$ и $i_2$ попадают в идеал, пораждая элемент $d$.
Пораждающий элемент $i$ - НОД всех этих элементов.\\

\bd{Основная теорема Алгебры}\\
Поле комплексных чисел является алгебраически замкнутым, то есть в нем любой 
многочлен с комплексными коэффициентами раскладывается на минимальный 
множттель\\

\begin{center}
  \bd{Симметрические многочлены. Теорема Виета}
\end{center}

Пусть P - поле. Рассмотрим кольцо многочленов от n элементов\\
$F(x_1, x_4) = \sum \alpha x^{i_1}_1, x^{i_2}_2, x^{i_3}_3, x^{i_4}_4$\\
Каждое слогаемое называется мономом $x^{5}_1, x^{7}_2, x^{16}_3$ 
Общая степень = 28\\
Когда у нас одна переменная $x$, то мы можем мономы легко сравнивать. Но когда
переменных $x0$ - несколько, то упорядочить их можно многими способами.\\
Способ упорядочения, который принят в словарях - называется 
лексико-графическим.\\
Допустим на множестве мономов мы ввели линейное упорядочение:\\
Первое условие - самый большой моном в смысле этого упорядочения - старший
моном.\\
Второе условие - упорядочение должно быть таким, чтобы количество меньших или
старших было конечно.\\
Третье условие - Сначала мономы сравниваются по общей степени, а когда она
совпадает, то сравниваем по лексико-графически.\\

\bd{Определение}\\
Многочлен от n-переменных называется симметричным, если он не изменяется при 
любой перестановке входящих в него символов.\\

\bd{Теорема}\\
Любой симметрический многочлен может быть выражен через элементы симметрии.\\

\bd{Теорема Виета}\\
Пусть $x_1, x_2, ... x_n$ - корни многочлена $f(x)$\\
$f(x) = (x - x_1)(x - x_2) ... (x - x_n) = x^n = S_1 x^{n - 1} + S_2 x^{n - 2} +
(-1)^n S_n = 0$\\
Коэффициент при степени многочлена с точностью до знака - является элементом 
симметрии многочлена от его корней.\\
Комбинируя теорему Виета и основную теосему о симметричности многочленов, мы
можем не находить корни многочлена (вычислять некоторые функции от неизвестных 
нам корней).\\
Так как нам требуются изучать перестановки переменных - надо ввести обозначать и
терминалогию из теории групп и перестановок.\\
$S_n = n!$ S - множество всех перестановок.\\

\begin{displaymath}
\left( \begin{array}{lccr}
1 & 2 & ... & n \\
i_1 & i_2 & ... & i_n
\end{array}\right)
\end{displaymath}

\begin{center}
  \bd{Математические основы Криптографии}
\end{center}

4 математические идеи\\

\bd{Первая идея} - ключ/шифр должен быть выбран случайным образом, но все 
имеющиеся алгебраическое основание.\\
\bd{Вторая идея} - однонаправленная функция $f$ - однонаправлена, если 
$f(n) = m$ - вычислить легко, но язная $m$ - найти $n$ - трудно.\\
\bd{Третья идея} - Хэш функция - отображает переводящее сообщение произвольной
длины в сообщении фиксированной длины. Разных хэшей - $2^256$. Но один хэш имеет
бесконечно много сообщений. Это свойство не для одного хэша  строго не доказано.
Хэш криптографический, если в процессе его вычисления используется шифрование.\\
\bd{Четвертая идея} - анализ протоколов. Интерактивный алгоритм в котором 
примимают участие 2 или больше ПРОПУСТИЛ В ЛЕКЦИИ - называется протоколом.
Если используется шифрование, то протокол - криптографический. Все компьютерные
процессы - осуществляется посредством протокола.

\end{document}
