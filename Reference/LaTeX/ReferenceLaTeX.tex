\documentclass{article}

\usepackage[utf8]{inputenc} 
\usepackage{amsmath,amssymb}
\usepackage[russian]{babel}
\usepackage[pdftex]{color}
\usepackage{ulem}
\parindent=0pt
\newcommand{\bb}[1]{{\bfseries $\backslash$#1}} %команда \bb{текст} текст делает жирным и перед ним ставит \

\begin{document}

Команды \TeX чувствительны к регистру.\\
Все команды записываются после символа $\backslash$ и неразрывно.\\
Если в исходнике стоит один или более пробелов то на выходе будет 1 пробел.\\
Если в искоднике стоит один или более отступов строки, то на выходе начнется новый абзац с отступом.\\

Символы  \{ \}  $\backslash$ \% \_  \^{}  \~{}  \&  \$  \# просто так неупотребляются в тексте их можно употреблять \verb|\{ \}  $\backslash$ \% \_  \^{}  \~{}  \&  \$  \#|.\\


\% коментарий строки при этом игнорируется конец строки поэтому следущие слова будут написаны слитно с предидущими.\\

Символы < > | + = - лучше употреблять только в формулах.\\

После команды пробелы игнорируется, чтобы итого  не проиходило надо либо заключить команду в   $\backslash$команда$\backslash$   либо поставить после нее   $\backslash$команда\{ \}.\\

Класс   \bb{documentclass}\{параметр\} определяет какого типаа будет текст и как он будет оформлен.\\

Класс \bb{begin}\{параметр\} \bb{end}\{параметр\} между ними надо набирать текст если вводить текст после то он будет игнорироватсья.\\

Часть файла расположеннай мужду   $\backslash$documentclass\{\}   $\backslash$begin\{\} называеться {\bfseries ПРЕАМБУЛОЙ}.\\

{\bfseries ГРУППЫ} тоесть \{ \} ограничивают действия команды.\\
\{\} значит обязательный аргумент для команды.\\
{[ ]} значит необязательный аргумент для команд.\\

\bb{usepackage}\{имя\_пакета\} подлкючает пакеты.\\
\bb{bfseries} {\bfseries жирный} шрифт.\\
\bb{mdseries} отменяет жирный шрифт.\\

\bb{slshape} {\slshape наклонный} шриф.\\
\bb{upshape} отменяет наклонный шрифт.\\

\bb{parindent=ед.изм} абзац становится ед.изм.\\

{\bfseries МАТЕМАТИЧЕСКИЕ ФОРМУЛЫ.}\\
Матматические формулы в тексте записываются между \$формула\$ или $\backslash$(формула$\backslash$) или $\backslash$begin\{math\}формула$\backslash$end\{math\}. Если надо записать их отдельно от текста то $\backslash$[формула$\backslash$] или $\backslash$begin\{displaymath\}формула$\backslash$end\{displaymath\}. Для того чтобы формулы нумеровались $\backslash$begin\{equation\}формула$\backslash$end\{equation\} и можно отметить фомулу $\backslash$label\{метка\} и сослаться на нее в тексте с помощью $\backslash$ref\{метка\} или $\backslash$eqref\{метка\}.\\

Если нужен перечеркнутый символ, то надо перед командо или символом поставить $\backslash$not. 
{\bfseries \^{}} степени  если выражение состоит более чем из одного символа то надо взять их в \{\} скобки.\\
{\bfseries \_} индексы (если выражение состоит более чем из одного символа то надо взять их в \{\} скобки).\\
\bb{ge} $\ge$ больше или равно.\\
\bb{le} $\le$ меньше или равно.\\
\bb{subset} $\subset$ содержится.\\
\bb{supset} $\supset$ содержится перевернутое.\\
\bb{subseteq} $\subseteq$ содержится или равно.\\
\bb{supseteq} $\supseteq$ содержится или равно перевернутой.\\
\bb{vee} $\vee$ .\\
\bb{wedge} $\wedge$.\\
\bb{cup} $\cup$.\\
\bb{cap} $\cap$.\\
\bb{mid} $\mid$ вертикальная палка.\\
\bb{perp} $\perp$ перпендикулярно.\\
\bb{approx} $\approx$ приближенное значение.\\
\bb{ast} $\ast$ умножение звездочкой.\\
\bb{frac}\{числитель\}\{знаменатель\} $\frac{1*2}{1-3}$\\
\bb{left}(тип скобки) открывает скобку   \bb{right}(тип скобки) закрывает скобку\\
\bb{sqrt}[степень корня]\{подкореное значение\}\\
\bb{surd} $\surd$ знак корня.\\
{\bfseries ||} знак модуля\\
\bb{pm} $\pm$ плюс минус.\\
\bb{mp} $\mp$ минус плюс.\\
\bb{div} $\div$ делить.\\
\bb{setminus} $\setminus$ черта с лева на право.\\
\bb{times} $\times$ умножение крестом.\\
\bb{oplus} $\oplus$ плюс в круге.\\
\bb{odot} $\odot$ точка в круге.\\
\bb{otimes} $\otimes$ умноожение крестом в круге.\\
\bb{triangle} $\triangle$ треугольник.\\
\bb{diamondsuit} $\diamondsuit$ ромб.\\
\bb{Box} $\Box$ квадрат.\\
\bb{neg или lnot} $\lnot$ знак отрицатия.\\
\bb{ominus} $\ominus$ минус в круге.\\
\bb{oslash} $\oslash$ слэш и круге.\\
\bb{circ} $\circ$ маленький круглешек.\\
\bb{bullet} $\bullet$ закрашенный круглешок.\\
\bb{cdots} $\cdots$ многоточия по центру.\\ 
\bb{ldots} \ldots многоточия с низу.\\
\bb{vdots} $\vdots$ вертикальные 3 точки.\\
\bb{ddots} $\ddots$ диагональный 3 точки.\\ 
\bb{sin   $\backslash$cos} и тп элементарные функции записываются так.\\
\bb{measuredagle} $\measuredangle$ угол.\\
\bb{lim} $\lim$ предел.\\
\bb{sum} $\sum$ знак суммы.\\
\bb{infty} $\infty$ бесконечность.\\
\bb{to} $\to$ знак стремления.\\
\bb{forall} $\forall$ любой.\\
\bb{exists} $\exists$ существует.\\
\bb{mathbf}\{символы\} $\mathbf{ABCdef}$ делает симолы жирными.\\
\bb{mathrm} $\mathrm{ABCdef}$.\\
\bb{mathit} $\mathit{ABCdef}$.\\
\bb{mathnormal} $\mathnormal{ABCdef}$.\\
\bb{mathsf} $\mathsf{ABCdef}$.\\
\bb{mathtt} $\mathtt{ABCdef}$.\\
\bb{mathcal} $\mathcal{ABCdef}$.\\
\bb{mathfrac} $\mathfrak{ABCdef}$.\\
\bb{mathbb}\{символы\} $\mathbb{ABCdef}$ делает символы полужирными.\\
\bb{in} $\in$ пренадлежит.\\
\bb{ni} $\ni$ пренадлежит перевернутое.\\
\bb{notin} $\notin$ не пренадлежит.\\
\bb{neq} $\neq$ не равно.\\
\bb{overline}\{выражение\} $\overline{a+b=c}$ рисует линию над выражением.\\
\bb{underline}\{выражение\} $\underline{a+b=c}$ рисует линию под выражением.\\
\bb{overbrace}\{выражение\} $\overbrace{a+b=c}$ рисует над выражением фигурную скобку.\\
\bb{underbrace}\{выражение\} $\underbrace{a+b=c}$ рисует под выражением фигурную скобку.\\
\bb{!} $a\! b$ производитт отрицательный пробел.\\
\bb{,} $a\, b$ пробел.\\
\bb{:} $a\: b$ пробел.\\
\bb{;} $a\; b$ пробел.\\
\bb{} $a\ b$ пробел.\\
\bb{quad} $a\quad b$ пробел.\\
\bb{qquad} $a\qquad b$ пробел.\\
{\bfseries '} $y'$ штрих.\\
\bb{cdot} $\cdot$ знак точки.\\
\bb{vec} $\vec a$ знак вектора над одним сомволом.\\
\bb{overrightarrow}\{символы\} $\overrightarrow{AB}$ знак вектора над символами.\\
\bb{overleftarrow}\{символы\} $\overleftarrow{AB}$ знак вектора над символами со стрелкой в обратную сторону.\\
\bb{equiv} $a\equiv b$ эквивалентно.\\
\bb{bmod} $a \bmod b$ деление с остатком.\\
\bb{pmod}\{символы\} $x \equiv a \pmod{b}$ деление с остатком.\\
\bb{rightarrow или $\backslash$to} $\rightarrow$ стрелка напрвленая права.\\
\bb{leftarrow или $\backslash$gets} $\leftarrow$ стелка напрвленая в лево.\\
\bb{leftrightarrow} $\leftrightarrow$ стрелка напрвленная в лево и в вправо.\\
\bb{uparrow} $\uparrow$ стрелка направленная в верх.\\
\bb{downarrow} $\downarrow$ стрелка направленная вниз.\\
\bb{updownarrow} $\updownarrow$ стрелка напрвленная в верх и вниз.\\
\bb{nearrow} $\nearrow$ стрелка  по диагонали верх в прво.\\
\bb{searrow} $\searrow$ стрелка по диагонали вниз в прво.\\
\bb{swarrow} $\swarrow$ стрелка по диагонали вниз в лево.\\
\bb{nwarrow} $\nwarrow$ стрелка по диагонали вверз в лево.\\
\bb{leftleftarrows} $\leftleftarrows$ две стрелки наровлены в лево.\\
\bb{rightrightarrows} $\rightrightarrows$ две стрелки напрвлены в право.\\
\bb{leftrightarrows} $\leftrightarrows$ стрелка смотри в лево, другая в право.\\
\bb{rightleftarrows} $\rightleftarrows$ стрелака смотрм в право, другая в лево.\\
\bb{upuparrows} $\upuparrows$ две стрелки смотрят вврех.\\
\bb{downdownarrows} $\downdownarrows$ две стрелки смотрят вниз.\\
\bb{Leftarrow} $\Leftarrow$ стедовательно в левую сорону.\\
\bb{Rightarrow} $\Rightarrow$ следовательно в правую сторону.\\
\bb{Leftrightarrow} $\Leftrightarrow$ слеодовательно в леов и в право.\\
\bb{Uparrow} $\uparrow$ слодовательно в верх.\\
\bb{Downarrow} $\Downarrow$следовательно вниз.\\
\bb{Updownarrow} $\Updownarrow$ следовательно вниз и вверх.\\
\bb{mapsto} $\mapsto$ отображение.\\
\bb{rangle} $\rangle$.\\
\bb{langle} $\langle$.\\
\bb{binom}\{символы\}\{символы\} $\binom{a}{b}$ биномиальные коэффиценты.\\
\bb{int} $\int$ интеграл.\\
\bb{iint} $\iint$ 2 интеграла.\\
\bb{iiint} $\iiint$ 3 интеграла.\\
\bb{alpha} $\alpha$ греческая буква.\\
\bb{beta} $\beta$ греческая буква.\\
\bb{pi} $\pi$ греческая буква.\\
\bb{gamma} $\gamma$ греческая буква.\\
\bb{delta} $\delta$ греческая буква.\\
\bb{theta} $\theta$ греческая буква.\\
\bb{lambda} $\lambda$ греческая буква.\\
\bb{mu} $\mu$ греческая буква.\\
\bb{rho} $\rho$ греческая буква.\\
\bb{sigma} $\sigma$ греческая буква.\\
\bb{tau} $\tau$ греческая d
буква.\\
\bb{upsilon} $\upsilon$ греческая буква.\\
\bb{phi} $\phi$ греческая буква.\\
\bb{varphi} $\varphi$ греческая буква.\\
\bb{chi} $\chi$ греческая буква.\\
\bb{psi} $\psi$ греческая буква.\\
\bb{omega} $\omega$ греческая буква.\\
\bb{stackrel}\{символы с верху\}\{символы снизу\} $\stackrel{!}{=}$\\
\bb{prod} $\prod$ оператор умножения.\\
\bb{substack}\{символы\} \[\sum_{\substack{a<b \\ a+b}}\] или можно использовать окружение $\backslash$begin\{subarray\}\{спецификатор\} $\backslash$end\{subarray\}\\ \\
 $\backslash$begin\{array\}\{позиция в столбцах\} $\backslash$end\{array\} используется для построения матрицы примерно так же как и для таблиц. Можно использовтаь для верстки выражений таких как система неравенств. Так же как и в таблицах можно рисовать линии. Если формула занимает много места то вместо equation надо использовать окружение eqnarray(у каждой строки свой номер) и eqnarray*(номера не ставятся)\\
 \bb{setlength$\backslash$arraycolsep}\{рстояние\} задает растояние мужду столбцами в матрице, вызывается перед окружением.\\
 \bb{nonumber} не пишит номер для строки в которой вызван. \\
Пример:
\begin{verbatim}
\begin{displaymath}
\left\( \begin{array}{lcr}
a_{11} & a_{12} & a_{13} \\
a_{21} & a_{22} & a_{23} \\
a_{31} & a_{32} & a_{33}
\end{array}\right\)
\end{displaymath}
\end{verbatim}
\begin{displaymath}
\left( \begin{array}{lcr}
a_{11} & a_{12} & a_{13} \\
a_{21} & a_{22} & a_{23} \\
a_{31} & a_{32} & a_{33}
\end{array}\right)
\end{displaymath}
\\\\
\bb{phantom}\{символы\} делает сомволы невидимыми.\\
\bb{newtheorem}\{название\}[счетчик]\{текст\}[раздел] Аргумент название это краткое ключевое слово, используемоу для индинтификации теорему. Аргумент текст определяет настоящее название теоремы. Необязательные аргументы применяемы для немерации теоремы.\\

\bb{input}\{имя\_файла\}  вставляет исходник имя\_файла 

\bb{endinput} если в вызавать после $\backslash$input \{ \} то чтение файла будет осуществляться до вызова   $\backslash$endinput в этом файле.\\

\bb{include}\{имя\_файла\} вставляет начиная с новой страницы исходник имя\_файла ($\backslash$include нельзя употреблять в файле, который сам включается в текст с помощью   $\backslash$include)\\
\bb{incudeonly}\{имена\_файлов\} обявляется в преамбуле и $\backslash$include \{\ldots\} открывает файлы объявленные в   $\backslash$incudeonly \{\ldots\}\\

{\bfseries ЕДЕНИЦЫ ИЗМЕРЕНИЯ В \TeX}\\
pt = 0.35\\
pc = 12pt\\
mm = милиметр\\
cm = 10mm\\
in = 25.4mm\\
dd = пукт Дидо 1.07pt\\
cc = 12dd\\

{\bfseries ОКРУЖЕНИЕ} это фрагмет файла, начинающийся с теста   $\backslash$begin\{имяОкружения\} $\backslash$end\{имяОкружения\} (имеет только один аргумент).\\
\{окружение\\ 
{\bfseries itemize} подходить для простых списков.\\
{\bfseries enumerate} для нумерованных списков.\\
{\bfseries descriprtion} для описания.\\
{\bfseries flushleft} выравнивает текст по левому краю.\\
{\bfseries flushright} выравнивает текст по левому краю.\\
{\bfseries center} выравнивает текс по центру.\\
{\bfseries verbatim} тест будет написан так как он есть без какого либо вмешательсва \LaTeX~или можно написать в тексте \verb|\verb+текст+| где вместо знак <<+>> может быть любой знак в виде окраничителя кроме <<*>>, пробела и букв.\\

{\bfseries tabular} выстраивает таблицу.\\
\{таблицы \\

\bb{begin}\{tabular\}[\emph{позиция}]\{\emph{спецификация}\}\\ 

{\bfseries Спецификация} определяет формат таблицы:\\ 
{\bfseries l} для столбца текста выровненого слева.\\ 
{\bfseries r} для стоблца текста выровненого справа.\\
{\bfseries c} для централизованного текста.\\
{\bfseries p}\{\emph{ширина}\} ширина для столбца содержащего выровненый текст с переносом строк.\\
{\bfseries |} для вертикальной линии.\\ 
{\bfseries @} \{\ldots\} удаляет пробелы между столбцами и заменяет их тем что между собок.\\

{\bfseries Позиция} определяет вертикальное положение всего табличного окружения:\\
{\bfseries t} выравнивание во верхнему краю.\\
{\bfseries b} выравнивание по нижнему краю.\\
{\bfseries c} выравнивание по центру окржения.\\ 

Команды используемые {\bfseries внутри откружения tabular:}\\
{\bfseries \&} переходит к следующему столбцу.\\
\verb|\\| начинает новую строку.\\
\bb{hline} всавляет горизонтальную линию.\\
\bb{cline}\{j-i\} добовляет в таблицу неполные лини, где j и i номера столбцов над которыми должна проходить линия.\\ 
\bb{multicolumn}\{i\}\{спецификация\}\{текст\} объеденяет ячейки где i кол-во обедененых столбцов, спецификация спецификация, текст это выведеный текст в обедененые столбцы.\\

Если надо набирать таблцы которые будут переходить на другую страницу, то надо испольховать окружения longtable supertabular.\\
\}конец таблицы\\

{\bfseries Пллавающие обекты.}\\
$\backslash$begin\{figure\}[спецификации размещения]\\
$\backslash$begin\{table\}[спецификации размещения]\\
По умолчанию спецификация [tbr].\\
{\bfseries h} ставит объект здесь же.\\
{\bfseries t} ставит объект на верху страницы.\\
{\bfseries b} ставит объекст внизу страницы.\\
{\bfseries p} ставит объкт на специальной страницы для плавающих обектов.\\
{\bfseries !} нерасматривает большинство внутренних параметров, который могут предотвратить размещение этого обекта.\\
\bb{caption}[краткий вариант]\{текст заголовка таблицы или рисунка\}\\
%я еще не пробовал
\bb{listoffigures} печатает список рисунков.\\
\bb{listoftables} печатет список таблиц.\\
При помощи $\backslash$label $\backslash$ref можно делать в тексте сылки на плавающие объекты.\\
\bb{clearpage} немедленно размещает все плавающие обекты в очереди по месту и переходит на новую страницу. \bb{cleardoublepage} начинает новую правостороннею страницу.\\

\}конец окружений\\

\bb{@} означает конец предложения, ставится после точки для того чтобы не \LaTeX~не делал лишнего растояния.\\
$\backslash \backslash$ переход на новую строку.\\
\bb{newline} переход на новую строку.\\
\bb{newpage} переход на новую страницу.\\
\~{} делает ровно один пробел.\\
- одна черточка (дефис)\\
-- две черточки (короткое тире)\\
--- три черточки (длинное тирe)\\
$-$ \$черточка\$ (знак миуса)\\	
\bb{mbox}\{текст\} текст не будет переносится.\\
\bb{fbox}\{текст\} текст не будет переносится и вокруг появляется рамка.\\
\bb{today} сегодняшняя дата (\today).\\
\bb{underline}\{текст\} текс будет \underline{подчернкнут}.\\
\bb{emph}\{текст\} текст будет выделен \emph{курсивом}.\\
\bb{sout}\{text\} \sout{текст зачеркнут.} \bb{usepackage}\{ulem\} пакет для зачеркивания.\\
ЗАМЕЧАНИЕ: если вы используетет выделение в выделом тексте то будет обычный текст.\\
\verb|<<текст>>| обозначают кавычки <<текст>>.\\

%шрифт
\bb{textrm}\{текст\} \textrm{прямой шрифт}.\\
\bb{texttt}\{текст\} \texttt{пишущая машинка.}\\
\bb{textmd}\{текст\} \textmd{нормальный.}\\
\bb{textup}\{текст\} \textup{прямой шрифт.}\\
\bb{textsl}\{текст\} \textsl{наклонный шрифт.}\\
\bb{textsf}\{текст\} \textsf{без зачесек.}\\
\bb{textbf}\{текст\} \textbf{без зачечек.}\\
\bb{textit}\{текст\} \textit{полужирный.}\\
\bb{textsc}\{текст\} \textsc{капитель.}\\
\bb{textnormal}\{текст\} \textnormal{обычный.}\\

%размер
\bb{tiny} {\tiny крошечный}.\\
\bb{scriptsize} {\scriptsize очень маленький}.\\
\bb{footnotesize} {\footnotesize довольно маленький}.\\
\bb{small} {\small маленький}.\\
\bb{normalsize} {\normalsize нормальный}.\\
\bb{large} {\large большой}.\\
\bb{Large} {\Large еще больше}.\\
\bb{LARGE} {\LARGE очень большой}.\\
\bb{huge} {\huge огромный}.\\
\bb{Huge} {\Huge Громадный}.\\

%сылки
\bb{label}\{метка\} оставляет метку.\\
\bb{ref}\{метка\} заменяет вызов коменды на номер раздела, подраздела, абзаца, уравнения, илюстрации.\\
\bb{pageref}\{метка\} заменяет вызов команды на страницу метки.\\

\bb{protect} делает из хрупких команд номральные. Относится только к команде следущей сразуе за ней.\\

\bb{footnote}\{текст\_сноски\} образет сноску на экране.\\

\bb{newcommand}\{названиие новой команды\}[число обязательных аргументов]\{определение\} создает новую уоманду. Пример:
\begin{verbatim}
\newcommand{bb}[1]{{\{\bfseries $\backslash$#1}}
\end{verbatim} 
\bb{renewcommand}\{названиие новой команды\}[число обязательных аргументов]\{определение\} заеменяет существующею команду.\\
\bb{providecommand}\{названиие новой команды\}[число обязательных аргументов]\{определение\} если команда уже существует то ничгео не делает.\\

\bb{newenvironment}\{название\}[номер]\{начало\}\{конец\} создает новое окружение.\\
\bb{renewenvironment}\{название\}[номер]\{начало\}\{конец\} создает новое окружение даже если окржуение с таким названием уже существует.\\

\bb{ProvidesPackage}\{название пакета\} надо вызвать в отедльном файле с расширением .sty и после определить новые команды. И можно будет подключать как обычный пакет в файле.\\

\bb{par} эквивалент пустой строки.\\

\bb{linespread}\{коээфицент\} изменяет интревал между строк. По умолчанию коэффицент 1.\\
\bb{hspace}\{длина\} добовляет горизонтальный пробел.\\
\bb{stretch}\{n\} резиновый пробел.\\
\bb{vspace}\{длина\} длина между абзацами.\\

%цвета
Цвета. RGB принимает значения 0-1.\\
\bb{usepackage}[pdftex]\{color\} пакет использования цветов.\\
\bb{definecolor}\{name\}\{model\}\{spec\} определение нового имени цвета.\\
\bb{textcolor}[model]\{spec\}\{text\} \textcolor[rgb]{1,0,0}{задает цвет тексту, модель надо выбрать rgb.}\\
{\bb{color}[model]\{spec\}text\} {\color[rgb]{0,1,0}задает цвет тексту.}\\
\bb{colorbox}[model]\{spec\}\{lr-text\} \colorbox[rgb]{0,0,1}{помещает текст в цветную коробку.}\\
\bb{fcolorbox}[model]\{fr-spec\}\{spec\}\{lr-text\} \fcolorbox[rgb]{1,0,0}{0,1,1}{помещает текст в цветовую коробку и обводит бокс рамкой.}\\
\bb{pagecolor}[model]\{spec\} окрашеивает до тех этим цветом пока не будет встречена команда с другим цветом.\\



\end{document}
