\begin{center}
	{\bfseries{\huge Математический анализ}} \\
	\LARGE{(Второй семестр)} \\

\end{center}

	\large{Дифференцирование функции 1-й переменной} \\ 

\bk {Иследование монотонности функции с помощью производной} \\

\bd{Теорема №1} \\
Пусть $f(x)$ - непрерывана на отрезке $<a; b>$ и дифференцируема \\
$f`(x) \quad (a;b)$ \\
Если $\forall x \in (a,b) \quad f`(x) \ge 0 $ то $f(x) \nearrow \quad <a; b>$ \\ 
Если $\forall x \in (a,b) \quad f`(x) \le 0 $ то $f(x) \searrow \quad <a; b>$ \\ 

\bd{Доказательство} \\
Пусть $\forall x_{1}, x_{2} \in <a,b> \quad x_{1} < x_{2} $ \\
Запишем \kv{теорему Лагранжа о формуле конечных превращений} \\
$f(x_{2}) - f(x_{1}) = f`(c) \cdot (x_{2} - x_{1}) \ge 0 $ \\
Если $f`(c) \ge 0$ то $f(x_{2}) \ge f(x_{2})$ \\

\bd{Теорема №2} \\
Пусть $\forall x \in (a, b)$ \\
Если $\forall x \in (a, b) \quad f`(x) > 0$ то $f(x) \nearrow; (a; b)$ \\
Если $\forall x \in (a, b) \quad f`(x) < 0$ то $f(x) \searrow \quad (a; b)$ \\

\bd{Теорема} \\
$\forall x_{1}, x_{2} \in (a; b) \quad x_{1} < x_{2} \\
f(x_{2}) > f(x_{1}) \\
f(x_{2}) - f(x_{1}) = f`(c) \cdot (x_{2} - x_{1}) > 0$ \\

\bk {Исследование функции на экстремум с помощью производной} \\

\bd{Теорема №1} \\
Если: \\
	1) $f(x)$ дифференцируема в проколотой окрестности точки \kv{а}\\
	2) Непрерывна в точке \kv{a} \\
	3) При переходе через точку \kv{а} $f(x)$ меняется с - на +. \\
	То есть $\exists \delta > 0 \quad \forall x \in (a - \delta; a) \quad f`(x) < 0$ \\
		   $\exists \delta > 0 \quad \forall x \in (a; a + \delta) \quad f`(x) > 0$ \\
Тогда \kv{a} - точка строгого минимума. \\

\bd {Доказательство} \\
Пусть $x \in (a - \delta; a) \quad f(x) \cdot f(x) = f`(c) \cdot (x - a) > 0 $ \\  %Сверься со своей лекцией
$f(x) > f(a)$ в левой проколотой окрестности точки \kv{а} \\
$x \in (a; a + \delta) \\
f(x) - f(a) = f`(c) \cdot (x - a) > 0 ~~ f(x) > f(a) \\
\forall x \in (a - \delta; a+ \delta) \ \{a\} ~~ f(x) > f(a)$ \\
точка строгого минимума. \\

\bd {Теорема №2} \\
Пусть $f(x)$ в точке \kv{а} \\
$f^{(n)} (a) \neq 0 ~~ n > 1 \\
f^{(k)} (a) = 0 ~~ k = 1, 2, ..., n-1$ \\ 
Если \kv {n} - не четный, то в точке \kv {a} - нет экстремума. \\
Если \kv {n} - четный, то \kv {а} - точка экстремума. \\
При чем если $f^{(n)} (a) > 0$ то \kv {a} - точка строгого минимума. \\
Если  $f^{(n)} (a) < 0$ то \kv {a} - точка строгого максимума. \\

\bd {Доказательство} \\
Формула Тейлора ипользуется для доказательства в форме Пиано: \\
$f(x) - f(a) = \frac {f^{(n)} (a)} {n!} \cdot (x - a)^n + \alpha (x) \cdot (x - a)^n = (x - a)^n \left ( \frac {f^{(n)} (a)} { n!} + \alpha(x) \right) \\
\alpha (x) \to 0 ~~ x \to a$ \\
Выберем $\delta$ так, чтобы $\delta > 0 ~~ \forall x \in (a - \delta; a + \delta) > |d(x)| < \frac {f^{(n)}(a)}{2n!} \\
sign \left ( \frac {f^{(n)} (a)} {n!} + \alpha(x) \right) = sign (f^{(n)} (a)$ \\
Пусть \kv {r} - не четный. Если $(x - a)^n$ - меняет знак, то и f(x) - f(a) - меняет знак и нет экстремумума \\
Пусть \kv {n} - четный $f(x) - f(a) > 0$, то \kv {a} - точка строгого минимума. \\
Пусть \kv {n} - четный $f(x) - f(a) < 0$, то \kv {a} - точка строгого максимума. \\

\bk {Алгоритм нахождения наибольшего и наименьшего значения функции}
Дана функция $f(x)$ на промежутке $[a;b]$. Необходимо \\
1) найти все критические точки $x_{1}, x_{2}, ..., x_{n} \in [a; b]$ \\
2) вычислить значения критических точек и крайних точек отрезка 
$f(a), f(b); f(x_{1}), f(x_{2}), ..., f(x_{k}) ~~ k = 1, 2 ... n$ \\

$max \{f(a); f(b); f(x_{k}), k = 1, 2, ..., n\} = f - $ наибольший \\
$mim \{f(a); f(b); f(x_{k}), k = 1, 2, ..., n\} = f - $  наименьший\\