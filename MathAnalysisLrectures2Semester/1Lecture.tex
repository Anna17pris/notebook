\begin{title}
  Математический анализ.\\
  (Второй семестр)
\end{title}

\begin{title}
  Дифференцирование функции 1-й переменной.\\
  Иследование монотонности функции с помощью производной.
\end{title}

\begin{theorem}
  Пусть $f(x)$ - непрерывана на $<a; b>$ и имеет $f'(x)$ на $(a;b)$ то
  \\
  Если $\forall x \in (a,b) ~~ f'(x) \ge 0 $ то $f(x) \nearrow ~~ <a; b>$\\
  Если $\forall x \in (a,b) ~~ f'(x) \le 0 $ то $f(x) \searrow ~~ <a; b>$
\end{theorem}

\begin{proof}
  Пусть $\forall x_1, x_2 \in <a,b>$ и при этом $x_1 < x_2 $\\
  Запишем \kv{теорему Лагранжа о формуле конечных преращений.}\\
  $f(x_2) - f(x_1) = f'(c)(x_2 - x_1)$ очевидно что $x_2 - x_1 > 0 $\\
  Если $f'(c) \ge 0$ то $f(x_2) \ge f(x_1) \Rightarrow f(x)$ на $<a,b>$
  \kv{возрастает}.\\
  Если $f'(c) \le 0$ то $f(x_2) \le f(x_1) \Rightarrow f(x)$ на $<a,b>$
  \kv{убывает}.
\end{proof}

\begin{theorem}
  Пусть $f(x)$ - непрерывана на $<a; b>$ и имеет $f'(x)$ на $(a;b)$ то\\
  Если $\forall x \in (a, b) ~~ f'(x) > 0$ то $f(x)$ \kv{строго}
  $\nearrow (a; b)$\\
  Если $\forall x \in (a, b) ~~ f'(x) < 0$ то $f(x)$ \kv{строго}
  $\searrow (a; b)$
\end{theorem}

\begin{proof}
  Пусть $\forall x_1, x_2 \in <a,b>$ и при этом $x_1 < x_2 $\\
  Запишем \kv{теорему Лагранжа о формуле конечных преращений.}\\
  $f(x_2) - f(x_1) = f'(c)(x_2 - x_1)$ очевидно что $x_2 - x_1 > 0$\\
  Если $f'(c) > 0$ то $f(x_1) > f(x_2) \Rightarrow f(x)$ на $<a,b>$
  \kv{возрастает}.\\
  Если $f'(c) < 0$ то $f(x_2) < f(x_1) \Rightarrow f(x)$ на $<a,b>$
  \kv{убывает}.
\end{proof}

\begin{title}
  Исследование функции на экстремум с помощью производной.
\end{title}

\bd{Необходимые и достаточные условия экстремума.}\\
\begin{theorem}
  Если: \\
    1) $f(x)$ дифференцируема в проколотой окрестности точки $a$.\\
    2) Непрерывна в точке $a$.\\
    3) При переходе через точку $a$ $f(x)$ меняется с - на +. То есть\\
      $\exists \delta > 0 ~~ \forall x \in (a - \delta; a) ~~ f'(x) < 0$\\
      $\exists \delta > 0 ~~ \forall x \in (a; a + \delta) ~~ f'(x) > 0$\\
  Тогда $a$ - точка \kv{строгого минимума.}
\end{theorem}

\begin{proof}
  По формуле конечных приращений\\
  $\forall x \in (a - \delta; a) ~~ f(x) - f(a) = f'(c)(x - a)$ очевидно что
  $x - a < 0$ и \\ $f'(c) < 0$ $\Rightarrow f(x) > f(a)$\\
  $\forall x \in (a; a + \delta) ~~ f(x) - f(a) = f'(c)(x - a)$ очевидно что
  $x - a > 0$ и \\ $f'(c) > 0$ $\Rightarrow f(x) > f(a)$\\
  $\forall x \in (a - \delta; a+ \delta)\backslash \{a\} ~~ f(x) > f(a)$\\
  Тогда $a$ - точка \kv{строгого минимума.}
\end{proof}

\begin{theorem}
  Если: \\
    1) $f(x)$ дифференцируема в проколотой окрестности точки $a$.\\
    2) Непрерывна в точке $a$.\\
    3) При переходе через точку $a$ $f(x)$ меняется с + на -. То есть\\
      $\exists \delta > 0 ~~ \forall x \in (a - \delta; a) ~~ f'(x) > 0$\\
      $\exists \delta > 0 ~~ \forall x \in (a; a + \delta) ~~ f'(x) < 0$\\
  Тогда $a$ - точка \kv{строгого максимума.}
\end{theorem}

\begin{proof}
  По формуле конечных приращений\\
  $\forall x \in (a - \delta; a) ~~ f(x) - f(a) = f'(c)(x - a)$ очевидно что
  $x - a < 0$ и \\ $f'(c) > 0$ $\Rightarrow f(x) < f(a)$\\
  $\forall x \in (a; a + \delta) ~~ f(x) - f(a) = f'(c)(x - a)$ очевидно что
  $x - a > 0$ и \\ $f'(c) < 0$ $\Rightarrow f(x) < f(a)$\\
  $\forall x \in (a - \delta; a+ \delta)\backslash \{a\} ~~ f(x) < f(a)$\\
  Тогда $a$ - точка \kv{строгого максимума.}
\end{proof}

\begin{theorem}
  Пусть $f(x)$ в точке $a$\\
  $f^{(n)} (a) \neq 0 ~~ n > 1\\
  f^{(k)} (a) = 0 ~~ k = 1, 2, ..., n-1$\\
  Если $n$ - не четный, то в точке $a$ - нет экстремума. \\
  Если $n$ - четный, то $a$ - точка экстремума.\\
  При чем если $f^{(n)} (a) > 0$ то $a$ - точка строгого минимума.\\
  Если  $f^{(n)} (a) < 0$ то $a$ - точка строгого максимума.
\end{theorem}

\begin{proof}
  Формула Тейлора ипользуется для доказательства в форме Пиано: \\
  $f(x) - f(a) = \frac {f^{(n)} (a)} {n!} \cdot (x - a)^n + \alpha (x) \cdot (x - a)^n = (x - a)^n \left ( \frac {f^{(n)} (a)} { n!} + \alpha(x) \right) \\
  \alpha (x) \to 0 ~~ x \to a$ \\
  Выберем $\delta$ так, чтобы $\delta > 0 ~~ \forall x \in (a - \delta; a + \delta) > |d(x)| < \frac {f^{(n)}(a)}{2n!} \\
  sign \left ( \frac {f^{(n)} (a)} {n!} + \alpha(x) \right) = sign (f^{(n)} (a)$ \\
  Пусть $r$ - не четный. Если $(x - a)^n$ - меняет знак, то и f(x) - f(a) - меняет знак и нет экстремумума\\
  Пусть $n$ - четный $f(x) - f(a) > 0$, то $a$ - точка строгого минимума. \\
  Пусть $n$ - четный $f(x) - f(a) < 0$, то $a$ - точка строгого максимума.
\end{proof}

\bd{Алгоритм нахождения наибольшего и наименьшего значения функции.}
Дана функция $f(x)$ на промежутке $[a;b]$. Необходимо \\
1) найти все критические точки $x_{1}, x_{2}, ..., x_{n} \in [a; b]$ \\
2) вычислить значения критических точек и крайних точек отрезка
$f(a), f(b); f(x_{1}), f(x_{2}), ..., f(x_{k}) ~~ k = 1, 2 ... n$ \\

$max \{f(a); f(b); f(x_{k}), k = 1, 2, ..., n\} = f - $ \kv{наибольший}\\
$mim \{f(a); f(b); f(x_{k}), k = 1, 2, ..., n\} = f - $  \kv{наименьший}\\