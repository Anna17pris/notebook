\begin{center}
  \bd{Интегрирование функции одной переменной.}
\end{center}

\bd{Первообразная и неопределенный интеграл. Свойства:}\\
Пусть $F(x); f(x)$ пределена на $<a;b>$ F - непрерывна на $<a,b>$,
дифференцируема на $(a;b)$ и $\forall x \in (a; b) ~~ F'(x) = f(x)$, то $F(x)$
называют первообразной для $f(x)$ на $(a;b)$.\\

\bd{Свойства:}\\
1) Если F(x) - первообразная для $f(x)$ определена на $\Delta$ то $F(x) + C$
также первообразная к $f(x)$.
Доказательство: $(F(x) + C)' = F'(x) + 0 = f(x)$.\\

2) Если $F_{1}(x), F_{2}(x)$ обе первообразнозные для $f(x)$ на промежутке
$\Delta$, то $F_{1}(x) - F_{2}(x) = C$.\\
Доказательство: $(F_{1}(x) - F_{2}(x))' = F'_{1}(x) - F'_{2}(x) =
f(x) - f(x) \equiv 0$\\
$F_{1}(x) - F_{2}(x) = const$.\\

\bd{Определение:}\\
Совокупность всех первообразных для $f(x)$ на промежутке $\Delta$ называют
неопределенным интегралом от $f(x)$ и обозначают
\[\int f(x)dx\]
$f(x)$ - называют подынтервальной функцией.\\
$f(x)dx$ - Подинтегральным выражением.
\[\int f(x)dx = \not\{ F(x) + C \not\} \]
Где $F(x) + C$ - семейство функций.\\

1) Диффиренциал от неопределенного интеграла равен подинтергальному выражению:
\[d \left ( \int f(x)dx \right ) = f(x)dx \]
Доказательство:\\
\[d \left ( \int f(x)dx \right ) = d(F(x) + C) = dF(x) + 0 = F'(x)dx = f(x)dx\]

2)\[\int dF(x) = F(x) + C\]
Доказательство:\\
$\int dF(x) = \int F'(x)dx = F(x) + C$ \\

3) $a, b \in R$\\
\[\int (af(x) + bf(x)) = a \cdot \int f(x)dx + b \cdot \int g(x)dx\]
Доказательство в одну сторону:\\
$aF(x) + bG(x) \in \int (af(x) + bg(x))dx\\
aF'(x) + G'(x) = af(x) + bg(x)$ \\

\begin{center}
  \bd{Замена переменных (подстановка) в неопределенных интегралах.}\\
\end{center}
\bk{Теорема:}\\
Пусть $F(t)$ - первообразная для $f(x)$ на промежутке $T$ \\
$t = \varphi (x)$ непрерывна и дифференциируема на $\varphi(\delta) \subset T$ то\\
\[\int f(\varphi (x)) \varphi'(x)dx = F(\varphi (x)) + C\]
Доказательство:
\[(F(\varphi(x)) + C)' = F'(\varphi (x)) \cdot \varphi'(x) + 0 = f(\varphi(x)) \varphi'(x)\]\\