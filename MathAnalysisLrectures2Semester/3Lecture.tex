\begin{title}
  Интегрировнаие функции одной переменной.
\end{title}

\begin{title}[\Large]
  Первообразная и неопределенный интеграл и их cвойства.
\end{title}

\begin{defin}[первообразной]
  Пусть $F(x); f(x)$ определена на $<a;b>$, F - непрерывна на $<a,b>$ и
  дифференцируема на $(a;b)$ тогда $\forall x \in (a; b) ~~ F'(x) = f(x)$,
  то $F(x)$ называют \kv{первообразной} для $f(x)$ на $(a;b)$.
\end{defin}

\bd{Свойства:}\\
1) Если F(x) - первообразная для $f(x)$ и определена на $\Delta$ то $F(x) + C$
также первообразная к $f(x)$.\\
\begin{proof}
  $(F(x) + C)' = F'(x) + 0 = f(x)$.
\end{proof}

2) Если $F_{1}(x), F_{2}(x)$ обе первообразнозные для $f(x)$ на промежутке
$\Delta$, то $F_{1}(x) - F_{2}(x) = C$.\\
\begin{proof}
  $(F_{1}(x) - F_{2}(x))' = F'_{1}(x) - F'_{2}(x) = f(x) - f(x) \equiv 0$\\
  $F_{1}(x) - F_{2}(x) = const$.
\end{proof}

\begin{defin}[неопредленного интеграла]
  Совокупность всех первообразных для $f(x)$ на промежутке $<\Delta>$ называют
  \kv{неопределенным интегралом} от $f(x)$ и обозначают
  \[\int f(x)dx\]
  $f(x)$ - называют \kv{подинтервальной функцией.}\\
  $f(x)dx$ - \kv{Подинтегральным выражением.}
  \[\int f(x)dx = \cancel{\{} F(x) + C \cancel{\}}\]
  Где $F(x) + C$ - \kv{семейство функций.}
\end{defin}

1) Диффиренциал от неопределенного интеграла равен подинтергальному выражению:
\[d \left ( \int f(x)dx \right ) = f(x)dx \]
\begin{proof}
  \[d \left ( \int f(x)dx \right ) = d(F(x) + C) = d(F(x)) + 0 = F'(x)dx
  = f(x)dx\]
\end{proof}

2) \[\int d(F(x)) = F(x) + C\]
\begin{proof}
  \[\int d(F(x)) = \int F'(x)dx = F(x) + C\]
\end{proof}

3) $a, b \in R$\\
\[\int (af(x) + bf(x)) = a \int f(x)dx + b \int g(x)dx\]
\begin{proof}
  \begin{eqnarray*}
    aF(x) + bG(x) = \int (af(x) + bg(x))dx\\
    aF'(x) + bG'(x) = af(x) + bg(x)
  \end{eqnarray*}
\end{proof}

\begin{title}[\Large]
  Замена переменных (подстановка) в неопределенных интегралах.
\end{title}

\begin{theorem}
  Пусть $F(t)$ - первообразная для $f(t)$ на $<T>$ где $t = \varphi (x)$
  непрерывна и дифференциируема на $\varphi(\Delta) \subset T$ то\\
  \[\int f(\varphi (x)) \varphi'(x)dx = F(\varphi (x)) + C\]
\end{theorem}

\begin{proof}
  \[(F(\varphi(x)) + C)' = F'(\varphi (x)) \varphi'(x) + 0 = f(\varphi(x))
    \varphi'(x)\]\\
\end{proof}