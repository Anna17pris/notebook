\begin{title}
  Определение определенного интеграла. Необходимые условия существования.
\end{title}

$y = f(x)$ определена на $[a,b]$\\
$x_k \in [a,b]$ такое что $x_0 = a < x_1 < x_2 \ldots < x_k = b$ называют
\kv{разбиением} отрезка $[a,b]$ и обозначают $R[a,b] =
\{x_k | k = 1,2,3 \ldots n\}$\\
$\Delta_k = [x_{k-1}, x_k]$ тоже что и $\Delta{x_k} = x_k - x_{k-1}$\\
$\lambda(R) = max\Delta x_k$ где $1 \le k \le n$ \kv{мелкость разбиения}\\
$c_k \in \Delta_k ~~ \{c_k\} = \upsilon(R)$ \kv{выборка}\\
$\sum_{k=1}^{n} f(c_k)\Delta x_k$ \kv{интегральная сумма}\\

Для $f(x)$ на $[a,b]$ соотвецтвует разбиению R и выборке $\upsilon(R)$\\

\begin{defin}[определенного интеграла]
  Если существует число $I$ такое что для
  \begin{eqnarray*}
    \forall\epsilon>0 ~~ \exists\delta_{\epsilon}>0 ~~ \forall R[a,b] ~~~
    \lambda(R)<\delta ~~ \forall\upsilon(R) ~~~
    \left| \sum_{k=1}^{n} f(c_k)\Delta x_k - I \right| < \epsilon
  \end{eqnarray*}
  называется \kv{опрделенным интегралом} от функции $f$ на $[a,b]$ и обознают
  $$\int_{a}^{b} f(x)dx$$
\end{defin}

\begin{theorem}
  Необходимое существование определенного интеграла. Если для $f(x)$ на $[a,b]$
  $\exists \int_{a}^{b}$ то $f(x)$ ограничена на $[a,b]$.
\end{theorem}

\begin{proof}
  Предположим $f(x)$ ограничена в точке $a$ на первом отрезке любого разбиения
  на $[a,b]$\\
  Пусть $\epsilon = 1$ тогда $\exists \delta_1 > 0 ~~ \forall R[a,b] ~~
  \upsilon(R) < \delta_1 ~~ \forall \upsilon(R)$
  $I-1 < \sum_{k=1}{n}f(c_k)\Delta x_k < I+1$ зафиксируем $c_1, c_2, \ldots c_n$
  тогда $I-1 < \sum_{k=1}{n}f(c_k)\Delta x_k < I+1$
  Если для $f(x)$ на $[a,b]$ $\exists \int_{b}^{a}$ то $f(x)$
  является теоремой по Виету.
\end{proof}