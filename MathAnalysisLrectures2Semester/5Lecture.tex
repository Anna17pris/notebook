\begin{center}
  \bd{Пример Интегрирования рациональных дробей}\\
\end{center}

\[\int \frac{x^5 - 2x^4 + 5x^3 - 12x^2 + 16x}{x^4 + 2x^2 - 8x + 5} dx = \]\\
\bd{I}\\
Делим уголком числитель на знаменатель, дабы дробь привести к правильному виду\\
\[= \int (x - 2)dx + \int \frac{3x^3 - 5x + 10}{x^4 + 2x^2 - 8x + 5} = \]\\
\bd{II}\\
Понижаем степень икса, деля знаменатель на числитель, приводя уравнение к более
простому для решения виду.\\
\[= \int (x - 2)dx + \int \frac{3x^3 - 5x + 10}{(x-1)^2 (x^2+2x+5)} dx = \]\\
\bd{III}\\
%Не помню как это описывается
\[\frac{3x^3 - 5x + 10}{(x-1)^2 (x^2+2x+5)} = \frac{A}{x - 1} \frac{B}{(x - 1)^2}
\frac{C \cdot x + D}{x^2 + 2x + 5}\]\\
Затем находим \kv{A, D, C, D} и записываем уравнение, используя значения
переменных.\\
\bd{IV}\\
Дальше идет стандартное решение интегралов
\[\int (x - 2)dx + \int \frac{dx}{(x - 1)^2} + \int \frac{3x + 5}{x^2 + 2x + 5}
dx =\]\\ 
\[= \frac{x^2}{2} - 2x - \frac{1}{x-1} + \frac{3}{2} ln(x^2 + 2x + 5) + arctg
\frac{x + 1}{2} + C\]\\

\begin{center}
  \bd{Интегрирование иррациональных выражений}\\
\end{center}

$R(u, v, w ...)$, где \kv{R} - Рациональная дробь по отношению к каждому
элементу\\

\bd{I тип}\\
$\int R \left ( Cx; \left ( \frac{ax + b}{cx + d} \right )^{r_{1}};
\left ( \frac{ax + b}{cx + d} \right )^{r_{2}};
\left ( \frac{ax + b}{cx + d} \right )^{r_{m}} \right )dx$\\
$ad - bc \neq 0$, $r_{1}...r_{m}$ - рациональные числа.
$r_{i} = \frac{p_{i}}{q_{i}}$\\

\bd{II тип}\\
$\int R \left (x; \sqrt{ax^2 + bx + c} \right )dx$\\
$a \neq 0$\\
1) $a > 0 ~~~ \sqrt{ax^2 + bx + c} = \pm \sqrt{a}x \pm t$\\
2) $c > 0 ~~~ \sqrt{ax^2 + bx + c} = \pm xt \pm \sqrt{c}$\\
2) $D > 0 ~~~ \sqrt{ax^2 + bx + c} = \pm t (x - x_{0})$\\

\bd{III тип}\\
$\int x^m \cdot (ax^n + b)^p dx$\\
$m, n, p \in Q ~~~ a, b \in R$\\
1) $p \in Z ~~~ x = t^q$\\
2) $\frac{m+1}{n} \in Z ~~~ ax^n + b = t^q$\\
3)$\frac{m+1}{n} + p \in Z ~~~ a + \frac{b}{x^n} = t^q$\\
 