\begin{center}
  \bd{Пример Интегрирования рациональных дробей}\\
\end{center}

\[\int \frac{x^5 - 2x^4 + 5x^3 - 12x^2 + 16x}{x^4 + 2x^2 - 8x + 5} dx = \]

\bd{I}\\
Делим уголком числитель на знаменатель, чтобы дробь привести к правильному виду
\[= \int (x - 2)dx + \int \frac{3x^3 - 5x + 10}{x^4 + 2x^2 - 8x + 5} = \]

\bd{II}\\
Раскладываем знаменатель на множетели.
\[= \int (x - 2)dx + \int \frac{3x^3 - 5x + 10}{(x-1)^2 (x^2+2x+5)} dx = \]

\bd{III}\\
Раскладываем дробь на простые дроби с неопрделенными коэффицентами.
\[
  \frac{3x^3 - 5x + 10}{(x-1)^2 (x^2+2x+5)} = \frac{A}{x - 1} +
  \frac{B}{(x - 1)^2} + \frac{C \cdot x + D}{x^2 + 2x + 5}
\]

\bd{IV}\\
Приводим правую часть к общему знаменателю и убираем знаменатель, раскрываем
скобки и записываем в матрицу построчно кэффеценты перед иксами с одинаковой
степенью.\\

\bd{V}\\
Дальше идет стандартное решение интегралов
\begin{eqnarray*}
  \int (x - 2)dx + \int \frac{dx}{(x - 1)^2} + \int \frac{3x + 5}{x^2 + 2x + 5}
    dx = \\
  =\frac{x^2}{2} - 2x - \frac{1}{x-1} + \frac{3}{2} ln(x^2 + 2x + 5) + \arctg
    \frac{x + 1}{2} + C
\end{eqnarray*}

\begin{title}[\Large]
  Интегрирование иррациональных выражений.
\end{title}

$R(u, v, w \ldots)$, где $R$ - Рациональная дробь по отношению к каждому
элементу\\

\bd{I тип}\\
\[
\int R \left( Cx; \left( \frac{ax + b}{cx + d} \right)^{r_1};
\left( \frac{ax + b}{cx + d} \right)^{r_2} \cdots
\left( \frac{ax + b}{cx + d} \right)^{r_m} \right)dx
\]
$ad - bc \neq 0$, $r_1 \ldots r_m\in Q ~~ r_i = \frac{p_i}{n_i}$
$\left( \frac{ax + b}{cx + d} \right) = t^q$ где $q$ наименьший общий делитель
  $n_i$\\

\bd{II тип}\\
\[\int R \left(x; \sqrt{ax^2 + bx + c} \right)dx\]
$a \neq 0$\\
1) $a > 0 ~~~ \sqrt{ax^2 + bx + c} = \pm \sqrt{a}x \pm t$\\
2) $c > 0 ~~~ \sqrt{ax^2 + bx + c} = \pm xt \pm \sqrt{c}$\\
2) $D > 0 ~~~ \sqrt{ax^2 + bx + c} = \pm t (x - x_0)$\\

\bd{III тип}\\
\[\int x^m \cdot (ax^n + b)^p dx\]
$m, n, p \in Q ~~~ a, b \in R$\\
1) $p \in Z ~~~ x = t^q$ где $q$ наименьший общий знаменатель $m$ и $n$.\\
2) $\frac{m+1}{n} \in Z ~~~ ax^n + b = t^q$ где $q$ знаменатель $p$.\\
3) $\frac{m+1}{n} + p \in Z ~~~ a + \frac{b}{x^n} = t^q$ где $q$ знаменатель
  $p$.\\